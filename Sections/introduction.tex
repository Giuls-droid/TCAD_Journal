\documentclass[../TCASII_jrnl.tex]{subfiles}
\graphicspath{{\subfix{images/}}}

\begin{document}
standard CMOS scaling is approaching its technological limits, and 3D integration is considered one of its most
promising alternatives \cite{mybib:beyne20063d}. With respect to a standard 2D design, die stacking is expected to reduce
overall wire length, thus bringing power, performance and area (PPA) improvements \cite{mybib:black2006stacking}. Stacked
3D-ICs can be manufactured using different process technologies, with the most common being Through Silicon Vias (TSV).
However, a tighter pitch of 3D interconnects is required to further boost these advantages. This can be achieved with
Face-to-Face (F2F) bonded 3D-ICs. Leveraging the Hybrid Bonding (HB) process \cite{mybib:katti20093d}, 3D system
integration with CuPads pitch below 1$\mu$m can be achieved \cite{mybib:HBref}. Subsequently, fine-grain
functional system partitioning with no area overhead for Die-to-Die connectivity is enabled, yielding better wire length
savings and associated benefits. Despite the progress in the processing technology, the power delivery for 3D-ICs is
still known to be a complex problem \cite{mybib:deliveryref1}, concerning both the inter-die and on-die power
delivery. While these issues are shared among all types of 3D-ICs, more options for power delivery are available for
more mature integration solutions. In fact, in typical designs, TSVs are used to connect the two dies, and they can be
used also for power and ground (P/G) nets \cite{mybib:TSVref2}, at the expenses of the die area. Leveraging the
fine pitch of 3D interconnect that is achieved by HB, stacking the dies F2F is a potential solution to the inter-die
power delivery issue, as the same bonding pads used for signals can be used to deliver power to the die on the top with
no area overhead. These improvements can be further boosted by minimizing also the impact of the on-die PDN. This can be
obtained by combining two innovative process technologies: Buried Power Rail (BPR) and Backside PDN (BS-PDN)
\cite{mybib:bprref}\cite{mybib:backsidePDN}. Both solutions have already proved very promising in a standard
2D design, but have yet to be implemented in a 3D scenario. In fact, a full PDN specifically designed for 3D F2F stacked
ICs has not been designed nor assessed from the IR-drop perspective. One example is mentioned in
\cite{mybib:monolithic}; however only the impact of the metal resources is analyzed and as a means to illustrate
the benefit of another type of 3D integration. This work aims at bridging these gaps by proposing a flexible 3D-PDN
design for F2F bonded IC, including backside processing and a 3D aware rail analysis flow to assess its performance.
\end{document}

\bibliographystyle{./bibtex/bib/IEEEtran}
\bibliography{./bibtex/bib/IEEEabrv,./bibtex/bib/TCASII_JournalBib.bib}